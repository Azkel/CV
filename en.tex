%% start of file `template.tex'.
%% Copyright 2006-2013 Xavier Danaux (xdanaux@gmail.com).
%
% This work may be distributed and/or modified under the
% conditions of the LaTeX Project Public License version 1.3c,
% available at http://www.latex-project.org/lppl/.


\documentclass[11pt,a4paper,sans]{moderncv}        % possible options include font size ('10pt', '11pt' and '12pt'), paper size ('a4paper', 'letterpaper', 'a5paper', 'legalpaper', 'executivepaper' and 'landscape') and font family ('sans' and 'roman')

% moderncv themes
\moderncvstyle{casual}                             % style options are 'casual' (default), 'classic', 'oldstyle' and 'banking'
\moderncvcolor{blue}                               % color options 'blue' (default), 'orange', 'green', 'red', 'purple', 'grey' and 'black'
%\renewcommand{\familydefault}{\sfdefault}         % to set the default font; use '\sfdefault' for the default sans serif font, '\rmdefault' for the default roman one, or any tex font name
%\nopagenumbers{}                                  % uncomment to suppress automatic page numbering for CVs longer than one page

% character encoding
\usepackage[utf8]{inputenc}                       % if you are not using xelatex ou lualatex, replace by the encoding you are using
%\usepackage{CJKutf8}                              % if you need to use CJK to typeset your resume in Chinese, Japanese or Korean

% adjust the page margins
\usepackage[scale=0.8]{geometry}
%\setlength{\hintscolumnwidth}{3cm}                % if you want to change the width of the column with the dates
%\setlength{\makecvtitlenamewidth}{10cm}           % for the 'classic' style, if you want to force the width allocated to your name and avoid line breaks. be careful though, the length is normally calculated to avoid any overlap with your personal info; use this at your own typographical risks...

% personal data
\name{Michał}{Smyk}
\title{Curriculum vitae}                               % optional, remove / comment the line if not wanted
\email{contact@smyk.it}                               % optional, remove / comment the line if not wanted
\homepage{www.smyk.it}                         % optional, remove / comment the line if not wanted
\photo[64pt][0.4pt]{picture}                       % optional, remove / comment the line if not wanted; '64pt' is the height the picture must be resized to, 0.4pt is the thickness of the frame around it (put it to 0pt for no frame) and 'picture' is the name of the picture file

% to show numerical labels in the bibliography (default is to show no labels); only useful if you make citations in your resume
%\makeatletter
%\renewcommand*{\bibliographyitemlabel}{\@biblabel{\arabic{enumiv}}}
%\makeatother
%\renewcommand*{\bibliographyitemlabel}{[\arabic{enumiv}]}% CONSIDER REPLACING THE ABOVE BY THIS

% bibliography with mutiple entries
%\usepackage{multibib}
%\newcites{book,misc}{{Books},{Others}}
%----------------------------------------------------------------------------------
%            content
%----------------------------------------------------------------------------------
\begin{document}
%\begin{CJK*}{UTF8}{gbsn}                          % to typeset your resume in Chinese using CJK
%-----       resume       ---------------------------------------------------------
\makecvtitle
\section{Experience}
\subsection{Vocational}
\cventry{2020--current}{Lead Software Engineer}{\href{https://www.unit4.com/}{Unit4 Poland}}{Wrocław}{}{
Responsibilities: Working within Unit4 DevOps Practices department, responsible for improving DevOps-related processes within company. Previously financial domain lead for Unit4 ERP software.
}
\cventry{2018--2020}{Software Engineer}{\href{https://www.unit4.com/}{Unit4 Poland}}{Wrocław}{}{
Responsibilities: Project management \& development of Unit4 Higher Education software, with focus on CI and CD tooling. Domain lead for Platform teams since Oct 2019, area product owner for DevOps topics. \newline{} 
Back-end technologies: \textit{ASP.Net MVC 5}, \textit{ASP.NET Core}\newline{}
DevOps tools: \textit{TeamCity}, \textit{Octopus}, \textit{Azure DevOps}
}
\cventry{2016--2018}{.NET Developer}{\href{https://www.credit-suisse.com}{Credit Suisse}}{Wrocław}{}{
	Responsibilities: Full-stack web developer, IT Product Owner 
}
\cventry{2015--2016}{Junior .NET Developer}{\href{https://www.credit-suisse.com}{Credit Suisse}}{Wrocław}{}{}
\cventry{2014-- 2015}{Technician}{\href{https://pwr.edu.pl/}{Wrocław University of Technology}}{Wrocław}{}{
Responsibiities: Back-end web development, database development
}

\section{Education}
\cventry{2016--2017}{Master of Science}{\href{https://pwr.edu.pl/}{Wrocław University of Technology}}{Computer Science}{}{} % arguments 3 to 6 can be left empty
\cventry{2012--2015}{Bachelor of Science}{\href{http://pwr.edu.pl/}{Wrocław University of Technology}}{Computer Science}{}{}

\section{Certificates}
\cvlistitem{\href{https://smyk.it/files/ADA.pdf}{Microsoft Certified Azure Developer Associate} - 2020}
\cvlistitem{\href{https://smyk.it/files/PSMI.pdf}{Professional Scrum Master I} - 2018}
\cvlistitem{\href{https://smyk.it/files/MCSD.pdf}{Microsoft Certified Solution Developer - App Builder} - 2018}

\section{Languages}
\cvitemwithcomment{Polish}{Fluent}{Native speaker}
\cvitemwithcomment{English}{Fluent}{\href{http://smyk.it/files/CAE.jpg}{Certificate in Advanced English (C1)}}

\section{Visit my site!}
There's more stuff I'm doing described there, like conference talks, my open source contributions or weird video games that I am developing! 

% Publications from a BibTeX file without multibib
%  for numerical labels: \renewcommand{\bibliographyitemlabel}{\@biblabel{\arabic{enumiv}}}% CONSIDER MERGING WITH PREAMBLE PART
%  to redefine the heading string ("Publications"): \renewcommand{\refname}{Articles}
\nocite{*}
\bibliographystyle{plain}
\bibliography{publications}                        % 'publications' is the name of a BibTeX file

% Publications from a BibTeX file using the multibib package
%\section{Publications}
%\nocitebook{book1,book2}
%\bibliographystylebook{plain}
%\bibliographybook{publications}                   % 'publications' is the name of a BibTeX file
%\nocitemisc{misc1,misc2,misc3}
%\bibliographystylemisc{plain}
%\bibliographymisc{publications}                   % 'publications' is the name of a BibTeX file
\end{document}


%% end of file `template.tex'.

