%% start of file `template.tex'.
%% Copyright 2006-2013 Xavier Danaux (xdanaux@gmail.com).
%
% This work may be distributed and/or modified under the
% conditions of the LaTeX Project Public License version 1.3c,
% available at http://www.latex-project.org/lppl/.


\documentclass[11pt,a4paper,sans]{moderncv}        % possible options include font size ('10pt', '11pt' and '12pt'), paper size ('a4paper', 'letterpaper', 'a5paper', 'legalpaper', 'executivepaper' and 'landscape') and font family ('sans' and 'roman')

% moderncv themes
\moderncvstyle{casual}                             % style options are 'casual' (default), 'classic', 'oldstyle' and 'banking'
\moderncvcolor{blue}                               % color options 'blue' (default), 'orange', 'green', 'red', 'purple', 'grey' and 'black'
%\renewcommand{\familydefault}{\sfdefault}         % to set the default font; use '\sfdefault' for the default sans serif font, '\rmdefault' for the default roman one, or any tex font name
%\nopagenumbers{}                                  % uncomment to suppress automatic page numbering for CVs longer than one page

% character encoding
\usepackage[utf8]{inputenc}                       % if you are not using xelatex ou lualatex, replace by the encoding you are using
%\usepackage{CJKutf8}                              % if you need to use CJK to typeset your resume in Chinese, Japanese or Korean

% adjust the page margins
\usepackage[scale=0.75]{geometry}
%\setlength{\hintscolumnwidth}{3cm}                % if you want to change the width of the column with the dates
%\setlength{\makecvtitlenamewidth}{10cm}           % for the 'classic' style, if you want to force the width allocated to your name and avoid line breaks. be careful though, the length is normally calculated to avoid any overlap with your personal info; use this at your own typographical risks...

% personal data
\name{Michał}{Smyk}
\title{Curriculum vitae}                               % optional, remove / comment the line if not wanted
%\address{Wrocław}{postcode city}{Poland}% optional, remove / comment the line if not wanted; the "postcode city" and and "country" arguments can be omitted or provided empty
%\phone[mobile]{+48~(234)~567~890}                   % optional, remove / comment the line if not wanted
\email{kontakt@smyk.it}                               % optional, remove / comment the line if not wanted
\homepage{www.smyk.it}                         % optional, remove / comment the line if not wanted
\photo[64pt][0.4pt]{picture}                       % optional, remove / comment the line if not wanted; '64pt' is the height the picture must be resized to, 0.4pt is the thickness of the frame around it (put it to 0pt for no frame) and 'picture' is the name of the picture file

% to show numerical labels in the bibliography (default is to show no labels); only useful if you make citations in your resume
%\makeatletter
%\renewcommand*{\bibliographyitemlabel}{\@biblabel{\arabic{enumiv}}}
%\makeatother
%\renewcommand*{\bibliographyitemlabel}{[\arabic{enumiv}]}% CONSIDER REPLACING THE ABOVE BY THIS

% bibliography with mutiple entries
%\usepackage{multibib}
%\newcites{book,misc}{{Books},{Others}}
%----------------------------------------------------------------------------------
%            content
%----------------------------------------------------------------------------------
\begin{document}
%\begin{CJK*}{UTF8}{gbsn}                          % to typeset your resume in Chinese using CJK
%-----       resume       ---------------------------------------------------------
\makecvtitle

\section{Edukacja}
\cventry{2016--2017}{Magister}{Politechnika Wrocławska}{Wrocław}{\textit{5.0}}{Informatyka, Specjalność: Inteligentne Systemy Informacyjne}  % arguments 3 to 6 can be left empty
\cventry{2012--2015}{Inżynier}{Politechnika Wrocławska}{Wrocław}{\textit{5.0}}{Informatyka}

\section{Praca Magisterska}
\cvitem{tytuł}{\emph{Metody ekstrakcji tekstów na podstawie filmowych ścieżek dźwiękowych}}
\cvitem{promotor}{Prof. Ngoc Thanh Nguyen, Prof. dr hab. inż.}
\cvitem{opis}{Analiza systemów automatycznego rozpoznawania mowy z użyciem sieci neuronowych na potrzeby generowania napisów do filmów.}

\section{Doświadczenie}
\subsection{Zawodowe}
\cventry{2016--aktualnie}{.NET Developer}{Credit Suisse}{Wrocław}{}{Aplikacje webowe w języku C\# używając środowiska ASP.NET MVC 5, wykorzystując bazę danych Oracle \newline{}Odpowiedzialny za back-end w C\#, front-end w JavaScripcie i rozwój bazy danych.}
\cventry{2015--2016}{Junior .NET Developer}{Credit Suisse}{Wrocław}{}{Aplikacje webowe w języku C\# używając środowiska ASP.NET MVC 5, wykorzystując bazę danych Oracle \newline{}Odpowiedzialny za back-end w C\#, front-end w JavaScripcie i rozwój bazy danych.}
\cventry{2014--2015}{Technik}{Politechnika Wrocławska}{Wrocław}{}{Aplikacje webowe w języku C\# używając środowiska ASP.NET MVC 5, wykorzystując bazę danych MS SQL \newline{}Odpowiedzialny głównie za tworzenie serwisów danych przy użyciu środowisk WCF i SignalR, bazując na architekturze  skierowanej na usługi (\emph{Service Oriented Architecture}}
\subsection{Różne}
\cventry{2014--2016}{Prezes}{Akademickie Stowarzyszenie Informatyczne}{Wrocław}{}{Pełnienie funkcji prezesa w Akademickim Stowarzyszeniu Informatycznym - organizacji studenckiej na Politechnice Wrocławskiej }
\cventry{2016--aktualnie}{Organizator ds. kwestii technicznych}{NiuCon Team}{Wrocław}{}{Pełnienie funkcji organizatora ds. kwestii technicznych podczas konwentów poświęconych Kulturze Japonii i Grom Komputerowym organizowanym we Wrocławiu.}
\cventry{2013--2016}{Członek zaplecza technicznego}{NiuCon Team}{Wrocław}{}{Pełnienie funkcji zaplecza technicznego podczas konwentów poświęconych Kulturze Japonii i Grom Komputerowym organizowanym we Wrocławiu.}

\section{Języki}
\cvitemwithcomment{Polski}{Płynny}{Język ojczysty}
\cvitemwithcomment{Angielski}{Płynny}{\href{http://smyk.it/files/CAE.jpg}{Certyfikat na poziomie C1 - Certificate in Advanced English}}
    \cvitemwithcomment{Japoński}{Podstawowy}{}
    \cvitemwithcomment{Francuski}{Podstawowy}{}

\section{Umiejętności programistyczne}
\cvdoubleitem{Duża znajomość}{C\#, Python}{Środowiska}{ASP.NET}
\cvdoubleitem{Średnia znajomość}{PL/SQL, JavaScript}{Środowiska}{jQuery}
\cvdoubleitem{Podstawowa znajomość}{Java, C++} {}{}

\section{Zainteresowania}
\cvitem{Sztuczna Inteligencja}{Jestem zainteresowany technologiami opierającymi się na sztucznej inteligencji i rozwiązaniach Big Data. Prowadzę własne projekty w tych technologiach, głównie za pomocą języka Python. q}
\cvitem{Kultura Japonii}{Jestem zafascynowany kulturą tego kraju, chciałbym kiedyś go odwiedzić.}
\cvitem{Pływanie}{Moja ulubiona forma aktywnego wypoczynku}
\cvitem{Muzyka}{Interesuję się techniczną stroną tworzenia muzyki. Czasami gram na gitarze. Kiedyś grałem na perkusji, do czego chciałbym wrócić w przyszłości.}
\cvitem{Gotowanie}{Lubię gotować w wolnym czasie. Głównie uczę się gotować potrawy kuchni azjatyckich, jednak nie ograniczam się tylko do nich. Specjalizuję się w głównych daniach i ciastach.}

\section{Moje Portfolio}
\cvlistitem{Przykłady moich projektów nie związanych z działalnością zawodową dostępne są na moim koncie na GitHubie (\href{https://github.com/Azkel}{github.com/Azkel})}

\section{Certyfikaty}
\cvlistitem{\href{https://smyk.it/files/MCP.pdf}{Microsoft Certified Professional}}
\cvlistitem{\href{https://smyk.it/files/XSeriesBigData.pdf}{Big Data XSeries from UC BerkeleyX - edX Verified Certficiate}}
\cvlistitem{\href{https://smyk.it/files/BDAS.pdf}{CS100.1x: Introduction to Big Data with Apache Spark - edX Verified Certficiate}}
\cvlistitem{\href{https://smyk.it/files/SML.pdf}{CS190.1x: Scalable Machine Learning - edX Verified Certficiate}}


\section{Szkolenia}
\cvlistitem{\href{https://smyk.it/files/MVC5Pl.pdf}{Trening ASP.NET MVC 5 przeprowadzony przez certyfikowanego trenera Microsoftu- styczeń 2016}}


% Publications from a BibTeX file without multibib
%  for numerical labels: \renewcommand{\bibliographyitemlabel}{\@biblabel{\arabic{enumiv}}}% CONSIDER MERGING WITH PREAMBLE PART
%  to redefine the heading string ("Publications"): \renewcommand{\refname}{Articles}
\nocite{*}
\bibliographystyle{plain}
\bibliography{publications}                        % 'publications' is the name of a BibTeX file

% Publications from a BibTeX file using the multibib package
%\section{Publications}
%\nocitebook{book1,book2}
%\bibliographystylebook{plain}
%\bibliographybook{publications}                   % 'publications' is the name of a BibTeX file
%\nocitemisc{misc1,misc2,misc3}
%\bibliographystylemisc{plain}
%\bibliographymisc{publications}                   % 'publications' is the name of a BibTeX file
\end{document}


%% end of file `template.tex'.
